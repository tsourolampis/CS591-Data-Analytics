% Based on Aleksander Madry's template

\documentclass[10pt]{article}
\usepackage[T1]{fontenc}
\usepackage{amssymb}
\usepackage{amsmath}
\usepackage{graphicx}
\usepackage{algpseudocode}
\usepackage{algorithm}
\usepackage{tikz}
\usetikzlibrary{calc, shapes, backgrounds}


\usetikzlibrary{arrows}
\usepackage{subfigure}
\usepackage{stackrel}
\usepackage{blindtext}
%\usepackage{hyperref} 

\oddsidemargin=0.15in
\evensidemargin=0.15in
\topmargin=-.5in
\textheight=9in
\textwidth=6.25in

\usepackage[colorlinks=true,breaklinks,pdfpagemode=none,linkcolor=blue,citecolor=blue]{hyperref}
\usepackage{enumerate}

%\usepackage{enumitem}
%\setlist{itemsep=0mm}



%\usepackage[usenames,dvipsnames]{pstricks}
%\usepackage{epsfig}
\usepackage{amsmath,amsfonts,amssymb,bm}
%\usepackage{pst-grad} % For gradients
%\usepackage{pst-plot} % For axes


%% Enviroment definitions (add your own here)

\newtheorem{theorem}{Theorem}
\newtheorem{corollary}[theorem]{Corollary}
\newtheorem{lemma}[theorem]{Lemma}
\newtheorem{observation}[theorem]{Observation}
\newtheorem{proposition}[theorem]{Proposition}
\newtheorem{definition}[theorem]{Definition}
\newtheorem{claim}[theorem]{Claim}
\newtheorem{fact}[theorem]{Fact}

\newenvironment{proof}{\noindent{\bf Proof}\hspace*{1em}}{\qed\bigskip}

%% New commands (add your own here)

\newcommand{\eps}{\varepsilon}
\newcommand{\bbR}{\mathbb{R}}
\newcommand{\hv}{\hat{v}}
\newcommand{\hL}{\hat{L}}
\newcommand{\hlambda}{\hat{\lambda}}
\newcommand{\homega}{\hat{\omega}}
\newcommand{\hp}{\hat{p}}
\newcommand{\hW}{\hat{W}}
\newcommand{\cK}{\mathcal{K}}
\newcommand{\qed}{\rule{7pt}{7pt}}
\newcommand{\cF}{\mathcal{F}}

\begin{document}

   \noindent
   \begin{center}

   \hrulefill
   
   \vspace{5pt}
   
   \makebox[\textwidth]{ {\bf \href{http://people.seas.harvard.edu/~babis/cs591_bu_sp17.html}{CS 591: Data Analytics -- Spring 2017}} \hfill  January 23, 2016}
   \vspace{0pt}
   
   {\Large \hfill  Lecture 1\hfill}
   \vspace{5pt}
   
   \makebox[\textwidth]{ {\it Lecturer: Charalampos E. Tsourakakis \hfill Scribe: (Scribe name here)} }
   
   \vspace{-3pt}
   \hrulefill
   \end{center}


\section{How to create figures: Tikz}

\tikzset{
  head/.style = {fill = orange!90!blue,
                 label = center:\textsf{\Large H}},
  tail/.style = {fill = blue!70!yellow, text = black,
                 label = center:\textsf{\Large T}}
}
\begin{tikzpicture}[
    scale = 1.5, transform shape, thick,
    every node/.style = {draw, circle, minimum size = 10mm},
    grow = down,  % alignment of characters
    level 1/.style = {sibling distance=3cm},
    level 2/.style = {sibling distance=4cm}, 
    level 3/.style = {sibling distance=2cm}, 
    level distance = 1.25cm
  ]
  \node[fill = gray!40, shape = rectangle, rounded corners,
    minimum width = 6cm, font = \sffamily] {Coin flipping} 
  child { node[shape = circle split, draw, line width = 1pt,
          minimum size = 10mm, inner sep = 0mm, font = \sffamily\large,
          rotate=30] (Start)
          { \rotatebox{-30}{H} \nodepart{lower} \rotatebox{-30}{T}}
   child {   node [head] (A) {}
     child { node [head] (B) {}}
     child { node [tail] (C) {}}
   }
   child {   node [tail] (D) {}
     child { node [head] (E) {}}
     child { node [tail] (F) {}}
   }
  };

  % Filling the root (Start)
  \begin{scope}[on background layer, rotate=30]
    \fill[head] (Start.base) ([xshift = 0mm]Start.east) arc (0:180:5mm)
      -- cycle;
    \fill[tail] (Start.base) ([xshift = 0pt]Start.west) arc (180:360:5mm)
      -- cycle;
  \end{scope}

  % Labels
  \begin{scope}[nodes = {draw = none}]
    \path (Start) -- (A) node [near start, left]  {$0.5$};
    \path (A)     -- (B) node [near start, left]  {$0.5$};
    \path (A)     -- (C) node [near start, right] {$0.5$};
    \path (Start) -- (D) node [near start, right] {$0.5$};
    \path (D)     -- (E) node [near start, left]  {$0.5$};
    \path (D)     -- (F) node [near start, right] {$0.5$};
    \begin{scope}[nodes = {below = 11pt}]
      \node [name = X] at (B) {$0.25$};
      \node            at (C) {$0.25$};
      \node [name = Y] at (E) {$0.25$};
      \node            at (F) {$0.25$};
    \end{scope}
    \draw[densely dashed, rounded corners, thin]
      (X.south west) rectangle (Y.north east);
  \end{scope}
\end{tikzpicture} 


\section{Bayes rule and an application} 




\end{document}

